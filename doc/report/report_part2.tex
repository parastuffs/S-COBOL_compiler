\documentclass[a4paper,11pt]{article}
 
\usepackage[english]{babel}
\usepackage[utf8x]{inputenc}
\usepackage[T1]{fontenc}
\usepackage{graphicx}
\usepackage{geometry}
\geometry{
		body={170mm,230mm},
		left=25mm,top=25mm,
		headheight=25mm,headsep=7mm,
		marginparsep=4mm,
		marginparwidth=20mm,
		footnotesep=50mm
}
\usepackage{enumitem}
\setlist{noitemsep}
\usepackage{amsmath}
\usepackage{amssymb}
\usepackage{mathabx}
\usepackage{tikz}
\usetikzlibrary{automata,positioning,arrows}
\usepackage{verbatim}
\usepackage{longtable}
\usepackage{array}

\newcounter{rule}
\newcommand{\addRule}[2]{%
	\stepcounter{rule} [\arabic{rule}] & #1 & $\rightarrow$ & #2 \tabularnewline
}
 
\title{INFO-F-403 Introduction to Language Theory and Compilation\\Project Part 2: \textsc{S-COBOL} Parsing, Semantic Analysis and Code Generation.}
\author{\textsc{Delhaye} Quentin}
 
\begin{document}
\maketitle

%%%%%%%%
%%%%%%%%
\section{Introduction}
The aim of this second part was to build a syntax analyzer (parser), semantic analyzer and code generator (to LLVM IR).

Firstly, the grammar had to be made LL(1) compliant, then the parser could be build.

\section{Grammar}
The table~\ref{tab:grammar} present the rules of the grammar.

The following modifications were applied:
\begin{itemize}
	\item removal of the left recursion in the various rules.
		Each time a left recursion occured, the rule has been split
		as presented in table~\ref{tab:leftRec}.
	\item Fusion of the common left parts.
	\item Transformations to respect the priorities of the operators.
	\item Addition of the rules \texttt{4} and \texttt{6} so that
		\texttt{WORD} can be an \texttt{INTEGER}.
	\item Removal of the right recursion in the rules derived from
		the variable \texttt{<EXP\_EQUAL\_LR>}. Indeed, having a language
		accepting semantics like \texttt{a < b > c = d} makes no sense.
\end{itemize}

%\begin{table}[!h]
%	\centering
\begin{longtable}{lll}
	A & $\rightarrow$ & Aa \tabularnewline
	  & becomes & \tabularnewline
	A & $\rightarrow$ & aA\_LR \tabularnewline
	A\_LR & $\rightarrow$ & aA\_LR \tabularnewline
	& $\rightarrow$ & $\varepsilon$ \tabularnewline
	\caption{Example of the removal of the left recursion in rule A.}
	\label{tab:leftRec}
\end{longtable}
%\end{table}

\begin{longtable}{lll>{\raggedright}p{10cm}}
	\addRule{<PROGRAM>}{<IDENT><ENV><DATA><PROC>}
	\addRule{<IDENT>}{identification division <END\_INST> program-id. ID<END\_INST> author. <WORDS><END\_INST> date-written. <WORDS><END\_INST>}
	\addRule{<WORDS>}{ID <WORDS\_LR>}
	\addRule{}{INTEGER <WORDS\_LR}
	\addRule{<WORDS\_LR>}{ID <WORDS\_LR>}
	\addRule{}{INTEGER <WORDS\_LR}
	\addRule{}{$\varepsilon$}
	\addRule{<END\_INST>}{.\textbackslash n}
	\addRule{<ENV>}{environment division<END\_INST> configuration section<END\_INST> source-computer. <WORDS><END\_INST> object-computer. <WORDS><END\_INST>
	}
	\addRule{<DATA>}{data division<END\_INST> working-storage section<END\_INST> <VAR\_LIST>}
	\addRule{<VAR\_LIST>}{<VAR\_DECL> <VAR\_LIST>}
	\addRule{}{$\varepsilon$}
	\addRule{<VAR\_DECL>}{<LEVEL> ID pic IMAGE <VAR\_DECL\_TAIL>}
	\addRule{<VAR\_DECL\_TAIL>}{value INTEGER<END\_INST>}
	\addRule{}{<END\_INST>}
	\addRule{<LEVEL>}{INTEGER}
	\addRule{<PROC>}{procedure division<END\_INST> ID section<END\_INST> <LABELS> end program ID.}
	\addRule{<LABELS>}{<LABEL><END\_INST> <INSTRUCTION\_LIST> <LABELS\_LR>}
	\addRule{<LABELS\_LR>}{<LABEL><END\_INST> <INSTRUCTION\_LIST> <LABELS\_LR>}
	\addRule{}{$\varepsilon$}
	\addRule{<LABEL>}{ID}
	\addRule{<INSTRUCTION\_LIST>}{<INSTRUCTION> <INSTRUCTION\_LIST>}
	\addRule{}{$\varepsilon$}
	\addRule{<INSTRUCTION>}{<ASSIGNATION>}
	\addRule{}{<IF>}
	\addRule{}{<CALL>}
	\addRule{}{<READ>}
	\addRule{}{<WRITE>}
	\addRule{}{stop run<END\_INST>}
	\addRule{<ASSIGNATION>}{move <EXPRESSION> to ID<END\_INST>}
	\addRule{}{compute ID = <EXPRESSION><END\_INST>}
	\addRule{}{add <EXPRESSION> to ID<END\_INST>}
	\addRule{}{subtract <EXPRESSION> from ID<END\_INST>}
	\addRule{}{multiply <ASSIGN\_END><END\_INST>}
	\addRule{}{divide <ASSIGN\_END><END\_INST>}
	\addRule{<ASSIGN\_END>}{<EXPRESSION>,<EXPRESSION> giving ID}
	\addRule{<EXPRESSION>}{<EXP\_AND> <EXPRESSION\_LR>}
	\addRule{<EXPRESSION\_LR>}{or <EXP\_AND> <EXPRESSION\_LR>}
	\addRule{}{$\varepsilon$}
	\addRule{<EXP\_AND>}{<EXP\_EQUAL> <EXP\_AND\_LR>}
	\addRule{<EXP\_AND\_LR>}{and <EXP\_EQUAL> <EXP\_AND\_LR>}
	\addRule{}{$\varepsilon$}
	\addRule{<EXP\_EQUAL>}{<EXP\_ADD> <EXP\_EQUAL\_LR>}
	\addRule{<EXP\_EQUAL\_LR>}{= <EXP\_ADD>}%TODO talk about the removal of the right recursion in the report.
	\addRule{}{< <EXP\_ADD>}
	\addRule{}{> <EXP\_ADD>}
	\addRule{}{<= <EXP\_ADD>}
	\addRule{}{>= <EXP\_ADD>}
	\addRule{}{$\varepsilon$}
	\addRule{<EXP\_ADD>}{<EXP\_MULT> <EXP\_ADD\_LR>}
	\addRule{<EXP\_ADD\_LR>}{+ <EXP\_MULT> <EXP\_ADD\_LR>}
	\addRule{}{- <EXP\_MULT> <EXP\_ADD\_LR>}
	\addRule{}{$\varepsilon$}
	\addRule{<EXP\_MULT>}{<EXP\_NOT> <EXP\_MULT\_LR>}
	\addRule{<EXP\_MULT\_LR>}{* <EXP\_NOT> <EXP\_MULT\_LR>}
	\addRule{}{/ <EXP\_NOT> <EXP\_MULT\_LR>}
	\addRule{}{$\varepsilon$}
	\addRule{<EXP\_NOT>}{-<EXP\_NOT>}
	\addRule{}{not <EXP\_NOT>}
	\addRule{}{<EXP\_PARENTHESIS>}
	\addRule{<EXP\_PARENTHESIS>}{(<EXPRESSION>)}
	\addRule{}{<EXP\_TERM>}
	\addRule{<EXP\_TERM>}{ID}
	\addRule{}{INTEGER}
	\addRule{}{true}
	\addRule{}{false}
	\addRule{<IF>}{if <EXPRESSION> then <INSTRUCTION\_LIST> <IF\_END>}
	\addRule{<IF\_END>}{else <INSTRUCTION\_LIST> end-if}
	\addRule{}{end-if}
	\addRule{<CALL>}{perform ID <CALL\_TAIL>}
	\addRule{<CALL\_TAIL>}{until <EXPRESSION><END\_INST>}
	\addRule{}{<END\_INST>}
	\addRule{<READ>}{accept ID<END\_INST>}
	\addRule{<WRITE>}{display <WRITE\_TAIL>}
	\addRule{<WRITE\_TAIL>}{<EXPRESSION><END\_INST>}
	\addRule{}{STRING<END\_INST>}

	\caption{LL(1) grammar of the S-COBOL language.}
	\label{tab:grammar}
\end{longtable}

\begin{longtable}{l>{\raggedright}p{6cm}p{6cm}}
	Variable & First\textsuperscript{1} & Follow\textsuperscript{1} \\ \hline
	<PROGRAM> & identification & \\
	<IDENT> & identification & environment\\
	<WORDS> & ID, INTEGER & .\\
	<WORDS\_LR> & ID, INTEGER, $\varepsilon$ & .\\
	<END\_INST> & . & program-id, date-written, environment, configuration, source-computer, object-computer, data, working-storage, INTEGER, $\varepsilon$, ID, move, compute, add, substract, multiply, divide, if, perform, accept, display, stop\\
	<ENV> & environment & data\\
	<DATA> & data & procedure\\
	<VAR\_LIST> & INTEGER, $\varepsilon$ & procedure \\
	<VAR\_DECL> & INTEGER & INTEGER, $\varepsilon$ \\
	<VAR\_DECL\_TAIL> & value, . & INTEGER, $\varepsilon$ \\
	<LEVEL> & INTEGER & ID\\
	<PROC> & procedure & \\
	<LABELS> & ID & end\\
	<LABELS\_LR> & ID, $\varepsilon$ & end\\
	<LABEL> & ID & .\\
	<INSTRUCTION\_LIST> & move, compute, add, substract, multiply, divide, if, perform, accept, display, stop, $\varepsilon$ & ID, $\varepsilon$\\
	<INSTRUCTION> & move, compute, add, substract, multiply, divide, if, perform, accept, display, stop &  move, compute, add, substract, multiply, divide, if, perform, accept, display, stop, $\varepsilon$ \\
	<ASSIGNATION> & move, compute, add, substract, multiply, divide & move, compute, add, substract, multiply, divide, if, perform, accept, display, stop, $\varepsilon$ \\
	<ASSIGN\_END> & -, not, (, ID, INTEGER, true, false & .\\
	<EXPRESSION> & -, not, (, ID, INTEGER, true, false & to, ., from, , , giving, ), then\\
	<EXPRESSION\_LR> & or, $\varepsilon$ & to, ., from, , , giving, ), then\\
	<EXP\_AND> & -, not, (, ID, INTEGER, true, false & or, $\varepsilon$ \\
	<EXP\_AND\_LR> & and, $\varepsilon$ & or, $\varepsilon$ \\
	<EXP\_EQUAL> & -, not, (, ID, INTEGER, true, false & and, $\varepsilon$ \\
	<EXP\_EQUAL\_LR> & =, <, >, <=, >=, $\varepsilon$ & and, $\varepsilon$ \\
	<EXP\_ADD> & -, not, (, ID, INTEGER, true, false & =, <, >, <=, >=, $\varepsilon$ \\
	<EXP\_ADD\_LR> & +, -, $\varepsilon$ & =, <, >, <=, >=, $\varepsilon$ \\
	<EXP\_MULT> & -, not, (, ID, INTEGER, true, false & +, -, $\varepsilon$ \\
	<EXP\_MULT\_LR> & *, /, $\varepsilon$ & +, -, $\varepsilon$ \\
	<EXP\_NOT> & -, not, (, ID, INTEGER, true, false & *, /, $\varepsilon$ \\
	<EXP\_PARENTHESIS> & (, ID, INTEGER, true, false & *, /, $\varepsilon$ \\
	<EXP\_TERM> & ID, INTEGER, true, false & *, /, $\varepsilon$ \\
	<IF> & if & move, compute, add, substract, multiply, divide, if, perform, accept, display, stop, $\varepsilon$ \\
	<IF\_END> & else, end-if & move, compute, add, substract, multiply, divide, if, perform, accept, display, stop, $\varepsilon$ \\
	<CALL> & perform & move, compute, add, substract, multiply, divide, if, perform, accept, display, stop, $\varepsilon$ \\
	<CALL\_TAIL> & until, . & move, compute, add, substract, multiply, divide, if, perform, accept, display, stop, $\varepsilon$ \\
	<READ> & accept & move, compute, add, substract, multiply, divide, if, perform, accept, display, stop, $\varepsilon$ \\
	<WRITE> & display & move, compute, add, substract, multiply, divide, if, perform, accept, display, stop, $\varepsilon$ \\
	<WRITE\_TAIL> & STRING, -, not, (, ID, INTEGER, true, false & move, compute, add, substract, multiply, divide, if, perform, accept, display, stop, $\varepsilon$ \\
	\caption{First and Follow table.}
\end{longtable}

\newpage
\begin{longtable}{l||ccccccc}
	& identification & ID & . & environment & data & INTEGER & value\\
	<PROGRAM> & 1 & & & & & & \\
	<IDENT> & 2 & & & & & & \\
	<WORDS> &  & 3 & & & & & \\
	<WORDS\_LR> & & 4 & 7 & & & & \\
	<END\_INST> & & & 8 & & & & \\
	<ENV> & & & & 9 & & & \\
	<DATA> & & & & & 10 & & \\
	<VAR\_LIST> & & & & & 11 & & 12 \\
	<VAR\_DECL> & & & & & 11 & & \\
	<VAR\_DECL\_TAIL> & & & 15 & & & & 14 \\
	<LEVEL> & & & & & & 16 & \\
	<PROC> & & & & & & & \\
	<LABELS> & & 18 & & & & & \\
	<LABELS\_LR> & & 19 & & & & & \\
	<LABEL> & & 21 & & & & & \\
	<INSTRUCTION\_LIST> & & 23 & & & & & \\
	<INSTRUCTION> & & & & & & & \\
	<ASSIGNATION> & & & & & & & \\
	<ASSIGN\_END> & & 36 & & & & 36 & \\
	<EXPRESSION> & & 37 & & & & 37 & \\
	<EXPRESSION\_LR> & & & 39 & & & & \\
	<EXP\_AND> & & 40 & & & & 40 & \\
	<EXP\_AND\_LR> & & & & & & & \\
	<EXP\_EQUAL> & & 43 & & & & 43 & \\
	<EXP\_EQUAL\_LR> & & & & & & & \\
	<EXP\_ADD> & & 50 & & & & 50 & \\
	<EXP\_ADD\_LR> & & & & & & & \\
	<EXP\_MULT> & & 54 & & & & 54 & \\
	<EXP\_MULT\_LR> & & & & & & & \\
	2<EXP\_NOT> & & 60 & & & & 60 & \\
	<EXP\_PARENTHESIS> & & 61 & & & & 61 & \\
	<EXP\_TERM> & & 63 & & & & 64 & \\
	<IF> & & & & & & & \\
	<IF\_END> & & & & & & & \\
	<CALL> & & & & & & & \\
	<CALL\_TAIL> & & & & & & & \\
	<READ> & & & & & & & \\
	<WRITE> & & & & & & & \\
	<WRITE\_TAIL> & & 75 & & & & 75 & \\
	\caption{Action table (1/5).}
\end{longtable}

\newpage
\begin{longtable}{l||ccccccc}
	& procedure & move & compute & add & substract & multiply & divide \\
	<PROGRAM> & & & & & & & \\
	<IDENT> & & & & & & & \\
	<WORDS> & & & & & & & \\
	<WORDS\_LR> & & & & & & & \\
	<END\_INST> & & & & & & & \\
	<ENV> & & & & & & & \\
	<DATA> & & & & & & & \\
	<VAR\_LIST> & 12 & & & & & & \\
	<VAR\_DECL> & & & & & & & \\
	<VAR\_DECL\_TAIL> & & & & & & & \\
	<LEVEL> & & & & & & & \\
	<PROC> & 15 & & & & & & \\
	<LABELS> & & & & & & & \\
	<LABELS\_LR> & & & & & & & \\
	<LABEL> & & & & & & & \\
	<INSTRUCTION\_LIST> & & 22 & 22 & 22 & 22 & 22 & 22 \\
	<INSTRUCTION> & & 24 & 24 & 24 & 24 & 24 & 24 \\
	<ASSIGNATION> & & 30 & 31 & 32 & 33 & 34 & 35 \\
	<ASSIGN\_END> & & & & & & & \\
	<EXPRESSION> & & & & & & & \\
	<EXPRESSION\_LR> & & & & & & & \\
	<EXP\_AND> & & & & & & & \\
	<EXP\_AND\_LR> & & & & & & & \\
	<EXP\_EQUAL> & & & & & & & \\
	<EXP\_EQUAL\_LR> & & & & & & & \\
	<EXP\_ADD> & & & & & & & \\
	<EXP\_ADD\_LR> & & & & & & & \\
	<EXP\_MULT> & & & & & & & \\
	<EXP\_MULT\_LR> & & & & & & & \\
	<EXP\_NOT> & & & & & & & \\
	<EXP\_PARENTHESIS> & & & & & & & \\
	<EXP\_TERM> & & & & & & & \\
	<IF> & & & & & & & \\
	<IF\_END> & & & & & & & \\
	<CALL> & & & & & & & \\
	<CALL\_TAIL> & & & & & & & \\
	<READ> & & & & & & & \\
	<WRITE> & & & & & & & \\
	<WRITE\_TAIL> & & & & & & & \\
	\caption{Action table (2/5).}
\end{longtable}


\newpage
\begin{longtable}{l||ccccccccc}
	& perform & accept & display & stop & - & not & ( & true & false\\
	<PROGRAM> & & & & & & & & & \\
	<IDENT>& & & & & & & & & \\
	<WORDS> & & & & & & & & & \\
	<WORDS\_LR> & & & & & & & & & \\
	<END\_INST>& & & & & & & & & \\
	<ENV>& & & & & & & & & \\
	<DATA> & & & & & & & & & \\
	<VAR\_LIST>& & & & & & & & & \\
	<VAR\_DECL> & & & & & & & & & \\
	<VAR\_DECL\_TAIL>& & & & & & & & & \\
	<LEVEL>& & & & & & & & & \\
	<PROC>& & & & & & & & & \\
	<LABELS>& & & & & & & & & \\
	<LABELS\_LR> & & & & & & & & & \\
	<LABEL> & & & & & & & & & \\
	<INSTRUCTION\_LIST>& 22 & 22 & 22 & 22 & & & & & \\
	<INSTRUCTION>& 26 & 27 & 28 & 29 & & & & & \\
	<ASSIGNATION>& & & & & & & & & \\
	<ASSIGN\_END>& & & & & 36 & 36 & 36& 36 & 36 \\
	<EXPRESSION> & & & & & 37 & 37 & 37 & 37 & 37 \\
	<EXPRESSION\_LR>& & & & & & & & & \\
	<EXP\_AND> & & & & & 40 & 40 & 40 & 40 & 40 \\
	<EXP\_AND\_LR>& & & & & & & & & \\
	<EXP\_EQUAL> & & & & & 43 & 43 & 43 & 43 & 43 \\
	<EXP\_EQUAL\_LR>& & & & & & & & & \\
	<EXP\_ADD> & & & & & 50 & 50 & 50 & 50 & 50 \\
	<EXP\_ADD\_LR>& & & & & 52 & & & & \\
	<EXP\_MULT> & & & & & 54 & 54 & 54 & 54 & 54 \\
	<EXP\_MULT\_LR>& & & & & 57 & & & & \\
	<EXP\_NOT> & & & & & 58 & 59 & 60 & 60 & 60 \\
	<EXP\_PARENTHESIS> & & & & & & & 61 & 62 & 62 \\
	<EXP\_TERM> & & & & & & & & 65 & 66 \\
	<IF> & & & & & & & & & \\
	<IF\_END> & & & & & & & & & \\
	<CALL> & 70 & & & & & & & & \\
	<CALL\_TAIL>& & & & & & & & & \\
	<READ> & & 72 & & & & & & & \\
	<WRITE>& & & 73 & & & & & & \\
	<WRITE\_TAIL> & & & & & 75 & 75 & 75 & 75 & 75 \\
	\caption{Action table (3/5).}
\end{longtable}

\newpage
\begin{longtable}{l||cccccccccccccc}
	& or & and & = & < & > & <= & >= & * & / & if & else & end-if & + & until\\
	<PROGRAM> & & & & & & & & & & & & & & \\
	<IDENT> & & & & & & & & & & & & & & \\
	<WORDS>  & & & & & & & & & & & & & & \\
	<WORDS\_LR>  & & & & & & & & & & & & & & \\
	<END\_INST> & & & & & & & & & & & & & & \\
	<ENV> & & & & & & & & & & & & & & \\
	<DATA>  & & & & & & & & & & & & & & \\
	<VAR\_LIST> & & & & & & & & & & & & & & \\
	<VAR\_DECL>  & & & & & & & & & & & & & & \\
	<VAR\_DECL\_TAIL> & & & & & & & & & & & & & & \\
	<LEVEL> & & & & & & & & & & & & & & \\
	<PROC> & & & & & & & & & & & & & & \\
	<LABELS> & & & & & & & & & & & & & & \\
	<LABELS\_LR>  & & & & & & & & & & & & & & \\
	<LABEL>  & & & & & & & & & & & & & & \\
	<INSTRUCTION\_LIST> & & & & & & & & & & 22 & & & & \\
	<INSTRUCTION> & & & & & & & & & & 25 & & & & \\
	<ASSIGNATION> & & & & & & & & & & & & & & \\
	<ASSIGN\_END> & & & & & & & & & & & & & & \\
	<EXPRESSION>  & & & & & & & & & & & & & & \\
	<EXPRESSION\_LR> & 38 & & & & & & & & & & & & & \\
	<EXP\_AND>  & & & & & & & & & & & & &  &\\
	<EXP\_AND\_LR> & 42 & 41 & & & & & & & & & & & & \\
	<EXP\_EQUAL>  & & & & & & & & & & & & & & \\
	<EXP\_EQUAL\_LR> & & 49 & 44 & 45 & 46 & 47 & 48 & & & & & & & \\
	<EXP\_ADD>  & & & & & & & & & & & & & \\
	<EXP\_ADD\_LR> & & & 53 & 53 & 53 & 53 & 53 & & & & & & 51 & \\
	<EXP\_MULT>  & & & & & & & & & & & & & & \\
	<EXP\_MULT\_LR> & & & & & & & & 55 & 56 & & & & 57 & \\
	<EXP\_NOT>  & & & & & & & & & & & & & & \\
	<EXP\_PARENTHESIS>  & & & & & & & & & & & & & & \\
	<EXP\_TERM>  & & & & & & & & & & & & & & \\
	<IF>  & & & & & & & & & & 67 & & & & \\
	<IF\_END>  & & & & & & & & & & & 68 & 69 & & \\
	<CALL>  & & & & & & & & & & & & & & \\
	<CALL\_TAIL> & & & & & & & & & & & & & & 71 \\
	<READ>  & & & & & & & & & & & & & & \\
	<WRITE> & & & & & & & & & & & & & & \\
	<WRITE\_TAIL>  & & & & & & & & & & & & & & \\
	\caption{Action table (4/5).}
\end{longtable}


\newpage
\begin{longtable}{l||cccccccc}
	& until & STRING & , & to & from & giving & ) & then \\
	<PROGRAM> & & & & & & & & \\
	<IDENT> & & & & & & & & \\
	<WORDS>  & & & & & & & & \\
	<WORDS\_LR>  & & & & & & & & \\
	<END\_INST> & & & & & & & & \\
	<ENV> & & & & & & & & \\
	<DATA>  & & & & & & & & \\
	<VAR\_LIST> & & & & & & & & \\
	<VAR\_DECL>  & & & & & & & & \\
	<VAR\_DECL\_TAIL> & & & & & & & & \\
	<LEVEL> & & & & & & & & \\
	<PROC> & & & & & & & & \\
	<LABELS> & & & & & & & & \\
	<LABELS\_LR>  & & & & & & & & \\
	<LABEL>  & & & & & & & & \\
	<INSTRUCTION\_LIST> & & & & & & & & \\
	<INSTRUCTION> & & & & & & & & \\
	<ASSIGNATION> & & & & & & & & \\
	<ASSIGN\_END> & & & & & & & & \\
	<EXPRESSION>  & & & & & & & & \\
	<EXPRESSION\_LR> & & & 39 & 39 & 39 & 39 & 39 & 39 \\
	<EXP\_AND>  & & & & & & & & \\
	<EXP\_AND\_LR> & & & & & & & & \\
	<EXP\_EQUAL>  & & & & & & & & \\
	<EXP\_EQUAL\_LR> & & & & & & & & \\
	<EXP\_ADD>  & & & & & & & & \\
	<EXP\_ADD\_LR> & & & & & & & & \\
	<EXP\_MULT>  & & & & & & & & \\
	<EXP\_MULT\_LR> & & & & & & & & \\
	<EXP\_NOT>  & & & & & & & & \\
	<EXP\_PARENTHESIS>  & & & & & & & & \\
	<EXP\_TERM>  & & & & & & & & \\
	<IF>  & & & & & & & & \\
	<IF\_END>  & & & & & & & & \\
	<CALL>  & & & & & & & & \\
	<CALL\_TAIL> & 71 & & & & & & & \\
	<READ>  & & & & & & & & \\
	<WRITE> & & & & & & & & \\
	<WRITE\_TAIL> & & 76 & & & & & & \\
	\caption{Action table (5/5).}
\end{longtable}

%%%%%%%%%%%%%%%%%%%%%%%%%%%%%
\section{Implementation}
\subsection{\texttt{Parser} Class}
This class mainly handles the parsing and the semantic analysis.

Each rule of the grammar has been implemented in recursive methods.
The parser can be accessed from the outside via the \texttt{parse(String)}
method, which takes as argument the String to parse (typically the source
code).

\subsection{\texttt{RaiseError} Class}
Whenever a problematic situation is encountered, one of those object is created.
Its main purpose is to write on the error output the error message.

\subsection{\texttt{WarningError} Class}
Works just like \texttt{RaiseError}, but throw a warning instead of an error.

\subsection{\texttt{LLVMGenerator} Class}
This class has a method for every functionnality of the S-COBOL language
that may be translated into LLVM IR.

It writes the code in a String, and finally write it into a file
upon external sollicitation of \texttt{toFile(String)} (the argument being
the filename.

%%%%%%%%%%%%%%%%%%%%%%%%%%%%%%%%%%%%%%
\section{Semantic analysis}
The following particularities are verified:
\begin{itemize}
	\item every variable can not be assigned with a number longer than
		what its declaration authorizes (the length of a number being
		the number of digits composing it). To serve that purpose,
		the types of the variables are transmited to the parent nodes
		in the recursive tree.
	\item The call list is checked at the end of the parsing. If a section
		is called without having been declared, an error is raised. If a
		section is declared without having been called, a warning is
		called.
\end{itemize}

%%%%%%%%%%%%%%%%%%%%%%%%%%%%%%%%%%%%%%%%%%%%%%%%%%%%%%
\section{Limitation of the current implementation}
This section acts as a conclusion and presents what is left to be improved.

\begin{itemize}
	\item The \texttt{OR} and \texttt{AND} keywords are not generated in
		the \texttt{LLVM IR} code.
	\item The order of some variables in the \texttt{LLVM IR} generated
		code are in reverse order.
	\item If an error is raised during the parsing, the execution is not
		stoped.
\end{itemize}


\end{document}


